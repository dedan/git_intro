%
%  untitled
%
%  Created by Stephan Gabler on 2010-12-02.
%  Copyright (c) 2010 __MyCompanyName__. All rights reserved.
%
\documentclass[]{article}

% Use utf-8 encoding for foreign characters
\usepackage[utf8]{inputenc}

% Setup for fullpage use
\usepackage{fullpage}

% Uncomment some of the following if you use the features
%
% Running Headers and footers
%\usepackage{fancyhdr}

% Multipart figures
%\usepackage{subfigure}

% More symbols
%\usepackage{amsmath}
%\usepackage{amssymb}
%\usepackage{latexsym}

% better links
\usepackage[pdfborder={0 0 0}]{hyperref}

% Surround parts of graphics with box
\usepackage{boxedminipage}

% Package for including code in the document
\usepackage{listings}

% If you want to generate a toc for each chapter (use with book)
\usepackage{minitoc}

% This is now the recommended way for checking for PDFLaTeX:
\usepackage{ifpdf}

%\newif\ifpdf
%\ifx\pdfoutput\undefined
%\pdffalse % we are not running PDFLaTeX
%\else
%\pdfoutput=1 % we are running PDFLaTeX
%\pdftrue
%\fi

\ifpdf
\usepackage[pdftex]{graphicx}
\else
\usepackage{graphicx}
\fi
\title{Homework for git introduction}
\author{Stephan Gabler}

\date{\today}

\begin{document}

\ifpdf
\DeclareGraphicsExtensions{.pdf, .jpg, .tif}
\else
\DeclareGraphicsExtensions{.eps, .jpg}
\fi

\maketitle

\begin{abstract}
    This homework is just to foster what I told you because I think these things sound all quite abstract and will be forgotten immediately if not used once. It looks like a lot, but only because I tried to explain every step. It should not take you more then 10 minutes.
\end{abstract}

\section{Getting the Software and Account}

\subsection{Install MySysGit} % (fold)
\label{sg:sub:install_mysysgit}
This gives you the git commandline tools and also a very basic (but not too bad) GUI (Graphical User Interface). On the page \url{http://code.google.com/p/msysgit/} you can get information on the software and find different installers and installation instructions. But it should be enough to just directly download the installer from \url{http://code.google.com/p/msysgit/downloads/detail?name=Git-1.7.3.1-preview20101002.exe} and install it with standard settings.

% subsection install_mysysgit (end)


\subsection{Install the TortoiseGIT GUI (optional but recommended)} % (fold)
\label{sg:sub:install_the_tortoisegit_gui_optional_}
A much more convenient GUI and many tools which are similar to what I showed you during the talk. Get the software from \url{http://code.google.com/p/tortoisegit/} and install it on your computer. Maybe you need to restart afterwards. TortoiseGIT will integrate into you File Browser (Explorer) and you access it by the context menu (rightclick) as I showed you in the in the presentation.

\textbf{This software works only when MySysGit already installed!}
% subsection install_the_tortoisegit_gui_optional_ (end)


\subsection{Get a Github account} % (fold)
\label{sg:sub:get_a_github_account}

Get an account on \url{https://github.com/signup/free}. This is were the repositories are hosted and it also has many other useful tools like a wiki and a ticket system. It is for free and many open-source code is hosted there. When you finished the registration send me a link with you username so I can make you a collaborator of the homework project \href{mailto:stephan.gabler@gmail.com}{\nolinkurl{stephan.gabler@gmail.com}}. You need this to finish the next steps.
% subsection get_a_github_account (end)


\section{Working on the repository} % (fold)
\label{sg:sec:working_on_the_repository}

\subsection{Get a working copy of the repository} % (fold)
\label{sg:sub:get_a_working_copy_of_the_repository}

After I made you a collaborator on the homework repository you can \emph{clone} a working copy of it to your computer in order to work with it. So go to a folder where you would like to have the code in, right-click and select \emph{Git Clone}. The URL to the repository can be found on its Github page: \url{https://github.com/dedan/git_intro}. There you can choose from three different types of URLs (you see only two if you are not a collaborator yet.). Choose the HTTP version and paste it into the \emph{Git Clone} window.

If you have any problems with this, or later during your homework, pleas write it into the Wiki on the Github page of the repository and notify me by email.
% subsection get_a_working_copy_of_the_repository (end)


\subsection{Apply changes to the repository} % (fold)
\label{sg:sub:apply_changes_to_the_repository}
By now you should have a working copy of the \emph{git\_intro} repository on you computer. It contains also the presentation and other documents related to the presentation. Also a subfolder \emph{homework} in which I ask you to complete the following tasks:

\begin{enumerate}
    \item add a line (maybe your favorite proverb) to the file \texttt{.homework/all.txt}
    \item commit these changes to your working repository (always write a few words in the commit message, describing what changes you made).
    \item add a file called \texttt{yourname.txt} to the homework folder and write some text in it. 
    \item add the new file to the repository
    \item also commit these changes
    \item pull from github repository to get the latest changes of other poeple.
    \item push the changes to the repository on github
\end{enumerate}

One remark to the pushing. I learned that TortoiseGit seams to have a bug and always crashes when you call the pushing command (Git Sync from the context menu) from any place but the top-folder of your repository. So before you push, got to the folder \texttt{git\_intro} and call the push command from right-clicking on it. 

% subsection apply_changes_to_the_repository (end)


\subsection{Problems} % (fold)
\label{sg:sub:problems}

    If you have any problems and you think other people might also come up with the same problems, please document the problem \textbf{and} the solution in the wiki on github. If it is a very special problem which is not likely to be encountered by the others, just write me an email.

% subsection problems (end)


% section working_on_the_repository (end)




\end{document}
